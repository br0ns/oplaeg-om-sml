% xcolor=table is because it seems that beamer uses the xcolor package in a
% strange way and thus don't accept us giving arguments to the package.
\documentclass[slidestop,compress,mathserif, xcolor=table]{beamer}

\usepackage[T1]{fontenc}
\usepackage[utf8]{inputenc}
\usepackage[danish]{babel}

\newcommand{\mft}{\textbf{MyFaceTube$^{\scriptsize{\textcopyright}}$}}

% \usepackage{mdwtab}
% \usepackage{mathenv}
% \usepackage{amsfonts}
% \usepackage{amsmath}
% \usepackage{amssymb}
% \usepackage{amsthm}

\usepackage{ulem}
\usepackage{semantic}
\renewcommand{\ttdefault}{txtt} % Bedre typewriter font

% Use the NAT theme in uk (also possible in DK)
\usetheme[nat,dk, footstyle=low]{Frederiksberg}

\setbeamercovered{invisible}
% possible to shift back, so they are just invisible untill they should overlay
% \setbeamercovered{invisible}

% Extend figures into either left or right margin
% Ex: \begin{narrow}{-1in}{0in} .. \end{narrow} will place 1in into left margin
\newenvironment{narrow}[2]{%
  \begin{list}{}{%
  \setlength{\topsep}{0pt}%
  \setlength{\leftmargin}{#1}%
  \setlength{\rightmargin}{#2}%
  \setlength{\listparindent}{\parindent}%
  \setlength{\itemindent}{\parindent}%
  \setlength{\parsep}{\parskip}}%
\item[]}{\end{list}}

\usepackage[final]{listings}
\lstset{ %
language=ml,                % choose the language of the code
basicstyle=\ttfamily\tiny,        % the size of the fonts that are used for the code
basewidth=0.5em,
numbers=left,                   % where to put the line-numbers
numberstyle=\tiny,      % the size of the fonts that are used for the line-numbers
% stepnumber=2,                   % the step between two line-numbers. If it's 1 each line will be numbered
% numbersep=5pt,                  % how far the line-numbers are from the code
% backgroundcolor=\color{white},  % choose the background color. You must add \usepackage{color}
% showspaces=false,               % show spaces adding particular underscores
% showstringspaces=false,         % underline spaces within strings
% showtabs=false,                 % show tabs within strings adding particular underscores
% frame=single	                % adds a frame around the code
% tabsize=2,	                % sets default tabsize to 2 spaces
% captionpos=b,                   % sets the caption-position to bottom
% breaklines=true,                % sets automatic line breaking
% breakatwhitespace=false,        % sets if automatic breaks should only happen at whitespace
escapeinside={(@}{@)}          % if you want to add a comment within your code
}

\lstnewenvironment{sml}
{\lstset{xleftmargin=1em}} % starting code, ex a lstset that is more specific;
{} % ending code


\usepackage{tikz}
\usetikzlibrary{calc,shapes,arrows}
% below is for use of backgrounds (foregrounds are not used so they are not
% added to this specification).
\pgfdeclarelayer{background}
\pgfsetlayers{background,main}


\usepackage{subfigure}


% Write a short text to have that shown in the footer of slides other than the
% title slide.
\title[]{Brug SML til jeres systemudviklingsprojekt}
% A possible subslide.
\subtitle{Om at skrive større programmer med SML}


\author[Morten Brøns-Pedersen]
       {Morten Brøns-Pedersen {\tiny(f@ntast.dk)}}

% Only write DIKU in the footer of slides (except the title slide).
\institute[DIKU]{Datalogisk institut}

% Remove the date stamp from the footer of slides (except title slide) by giving
% it no short "text"
\date[]{10. februar 2011}

\begin{document}

\frame[plain]{\titlepage}

\begin{frame}
  \frametitle{Program}

  \begin{itemize}
  \item Introduktion (med eksempel).
  \item Kort om systemdesign.
  \item SML's modulsystem.
  \item Et par eksempler.
  \item Det \sout{kedelige} praktiske.
  \item Afrunding.
  \end{itemize}\vfill

  \textit{Diaser og kode kan findes på \url{http://github.com/mortenbp}.}
\end{frame}

\begin{frame}
  \frametitle{Introduktion}

  På kurset Introduktion på Programmering har I lært SML at kende. Men de
  problemer I har løst har været små og veldefinerede.\\[1em]

  Målet for i dag er at gøre jer i stand til
  \begin{block}{}
    \center At strukturere, og udvikle, et større program med SML.
  \end{block}\ \\

  Vi starter med et eksempel.
\end{frame}

\begin{frame}
  \frametitle{Eksempel: Persondatabase \quad (1/13)}

  Designovervejelser
  \begin{itemize}
  \item Hvad skal den kunne? Vi kan f.eks. lave hurtigt opslag af personer på
    navn eller CPR-nummer, men ikke umiddelbart begge.
  \item Hvor generisk skal databasen være? Er det bare et tilfælde af en mængde
    eller en afbildning, eller skal den være mere specialiseret?
  \item Hvad er formålet med databasen? Formålet skal afspejles i designet; \textit{godt
      design kendes på hvad der er udeladt.}
  \item Skal det være muligt at arbejde med flere databaser, eller findes der
    kun én enkelt?
  \end{itemize}
\end{frame}

\begin{frame}
  \frametitle{Eksempel: Persondatabase \quad (2/13)}

  \begin{block}{\mft}
    En database som holder styr på venskaber mellem personer. Venskaber er ikke
    nødvendigvis gensidige.
    \center Der findes kun én \mft!
  \end{block}

  \mft er \textit{ikke}
  \begin{itemize}
  \item En telefonbog.
  \item Et stamtræ.
  \item Et fotoalbum.
  \item ...
  \end{itemize}

  Ekstra funktionalitet kan implementeres som en ny database eller direkte i
  \mft. Endnu et valg at træffe!
\end{frame}

\begin{frame}
  \frametitle{Eksempel: Persondatabase \quad (3/13)}

  Hvilke beregninger og hvilke objekter består \mft af?

  \begin{itemize}
  \item Personer.
  \item Mængder (af personer).
  \item Opslag (fra personer til mængder af personer) --- Hvem er dine venner?
  \item Selve \mft.
  \end{itemize}

  Vi starter med signaturerne.
\end{frame}

\begin{frame}[fragile]
  \frametitle{Eksempel: Persondatabase \quad (4/13)}
  \begin{block}{}
    \lstinputlisting{kode/persondatabase/ORDNING.sig}
  \end{block}
  \only<2->{
    Husk at \texttt{order} er defineret i standardbiblioteket:
    \begin{block}{}
      \lstinputlisting[numbers=none]{kode/til-slides/order.sml}
    \end{block}
  }
\end{frame}

\begin{frame}[fragile]
  \frametitle{Eksempel: Persondatabase \quad (5/13)}
  \begin{block}{}
    \lstinputlisting{kode/persondatabase/PERSON.sig}
  \end{block}
\end{frame}

\begin{frame}[fragile]
  \frametitle{Eksempel: Persondatabase \quad (6/13)}
  \begin{block}{}
    \lstinputlisting{kode/persondatabase/ORDNET_MAENGDE.sig}
  \end{block}
\end{frame}

\begin{frame}[fragile]
  \frametitle{Eksempel: Persondatabase \quad (7/13)}
  \begin{block}{}
    \lstinputlisting{kode/persondatabase/ORDNET_OPSLAG.sig}
  \end{block}
\end{frame}

\begin{frame}[fragile]
  \frametitle{Eksempel: Persondatabase \quad (8/13)}
  \begin{block}{}
    \lstinputlisting{kode/persondatabase/MY_FACE_TUBE.sig}
  \end{block}
\end{frame}

\begin{frame}
  \frametitle{Eksempel: Persondatabase \quad (9/13)}
  Vi gennemgår ikke hele implementeringen her. Koden kan findes online på
  \url{http://github.com/mortenbp}.

  \begin{block}{}
    \lstinputlisting{kode/persondatabase/Person.sml}
  \end{block}
\end{frame}

\begin{frame}
  \frametitle{Eksempel: Persondatabase \quad (10/13)}
  \begin{block}{}
    \lstinputlisting[lastline=15]{kode/persondatabase/OrdnetMaengdeFn.sml}
  \end{block}
  \begin{block}{}
    \lstinputlisting[numbers=none]{kode/persondatabase/PersonMaengde.sml}
  \end{block}
  \begin{block}{}
    \lstinputlisting[numbers=none]{kode/persondatabase/PersonOpslag.sml}
  \end{block}
\end{frame}

\begin{frame}
  \frametitle{Eksempel: Persondatabase \quad (11/13)}
  One of a kind? --- Imperativ programmering med SML.
  \begin{block}{}
    \lstinputlisting[lastline=19]{kode/persondatabase/MyFaceTube.sml}
  \end{block}
\end{frame}

\begin{frame}[c]
  \frametitle{Eksempel: Persondatabase \quad (12/13)}
  \begin{center}
    \Huge{Live demo.}
  \end{center}
\end{frame}

\begin{frame}
  \frametitle{Eksempel: Persondatabase \quad (13/13)}

  Alternativ løsning: Lad personer holde styr på deres egne venskaber.

  \begin{description}
  \item[Pro] \mft bliver essentielt blot en mængde af personer --- vi behøver
    ikke implementere opslag.
  \item[Contra] Vores \texttt{Person.t} bliver mindre generel --- i mange
    situationer er det uinteressant at vide hvem der er venner med hvem.
  \end{description}

  \uncover<2->{
  \begin{block}{}
    \lstinputlisting[firstline=1,lastline=1,numbers=none]
    {kode/til-slides/persontyper.sml}
  \end{block}
}
  \uncover<3->{
    \begin{block}{}
      \lstinputlisting[firstline=2,lastline=2,numbers=none]
      {kode/til-slides/persontyper.sml}
    \end{block}
  }
\end{frame}

\begin{frame}
  \frametitle{Kort om systemdesign}
  Tommelfingerregel:
  \begin{block}{}
    Des mindre to dele af systemet har med hinanden at gøre, des mindre bør de
    vide om hinanden.
  \end{block}
  \begin{itemize}
  \item Opdel systemet i beregning og repræsentation. Det er især vigtigt i
    funktionel programmering.
  \item Konventioner omkring repræsentation og invarianter er OK inden for
    strukturer, \textit{ikke} mellem dem.
  \item Undgå så vidt muligt at gøre (data)typer synlige for det omgivende
    program (brug \texttt{:>}). Lav i stedet en struktur som modellerer
    objektet.
  \item Brug de objekter der passer! Altså; ingen lister hvor der burde være
    mængder.
  \end{itemize}
\end{frame}

\begin{frame}
  \frametitle{SML's modulsystem \quad (1/3)}

  Signaturer som dokumentation. Eksempel fra MyLib:
  \begin{block}{}
    \lstinputlisting[firstline=10,lastline=33]
    {kode/til-slides/Layout.sig}
  \end{block}
\end{frame}

\begin{frame}
  \frametitle{SML's modulsystem \quad (2/3)}
  Brug en enkelt struktur for hvert objekt der skal repræsenteres.

  \begin{block}{}
    \center Objektets type er \texttt{Objekt.t}.
  \end{block}\ \\

  \textbf{OBS:} Strider med standardbibliotekets konvention som er
  \texttt{Objekt.objekt}.
\end{frame}

\begin{frame}
  \frametitle{SML's modulsystem \quad (3/3)}
  Udvidelse af strukturer.
  \begin{itemize}
  \item Signaturer kan inkludere andre signaturer med \texttt{include}.
    \begin{block}{}
    \lstinputlisting[firstline=1,lastline=5]
    {kode/til-slides/List.sig}
    \end{block}
  \item Strukturer kan inkludere andre strukturer med \texttt{open}.
    \begin{block}{}
    \lstinputlisting[firstline=1,lastline=3]
    {kode/til-slides/List.sml}
    \end{block}
  \end{itemize}
\end{frame}

\begin{frame}
  \frametitle{Et par eksempler}
  \begin{itemize}
  \item Når \texttt{option} ikke er god nok: \texttt{Either.t}.
  \item Modellering af klasser med funktorer: \texttt{FlagFn}.
  \item Gratis funktioner med funktorer: \texttt{FoldbarFn}.
  \item Doven evaluering med memoisering: \texttt{Lazy.t}.
  \end{itemize}
\end{frame}

\begin{frame}
  \frametitle{Et par eksempler: \texttt{Either.t}}

  \begin{block}{}
    \lstinputlisting{/home/morten/kode/sml/mylib/Other/Either.sig}
  \end{block}
\end{frame}

\begin{frame}
  \frametitle{Et par eksempler: \texttt{Either.t}}

  \begin{block}{}
    \lstinputlisting{/home/morten/kode/sml/mylib/Other/Either.sml}
  \end{block}
  \begin{block}{}
    \lstinputlisting[firstline=30,lastline=31]
    {/home/morten/kode/sml/mylib/TopLevel.sml}
  \end{block}
\end{frame}

\begin{frame}
  \frametitle{Et par eksempler: \texttt{FlagFn}}
  \textbf{FARE!} Imperativ programmering.

  \begin{block}{}
    \lstinputlisting{kode/FLAG.sig}
  \end{block}
  \begin{block}{}
    \lstinputlisting{kode/FlagFn.sml}
  \end{block}
\end{frame}

\begin{frame}
  \frametitle{Et par eksempler: \texttt{FlagFn}}
  \textbf{FARE!} Imperativ programmering.

  \begin{block}{}
    \lstinputlisting{kode/MineFlag.sml}
  \end{block}
\end{frame}

\begin{frame}
  \frametitle{Et par eksempler: \texttt{FoldbarFn}}

  \begin{block}{}
    \lstinputlisting{kode/FOLDBAR.sig}
  \end{block}
  \begin{block}{}
    \lstinputlisting{kode/FoldbarFn.sml}
  \end{block}
\end{frame}

\begin{frame}
  \frametitle{Et par eksempler: \texttt{FoldbarFn}}

  Mange ting er \texttt{FOLBAR}e:
  \begin{itemize}
  \item Opslag.
  \item Træer.
  \item Lister.
  \item Grafer.
  \item Filer.
  \item ...
  \end{itemize}
  De seks funktioner er nu "`gratis"' for hver slags objekt.

  \begin{block}{}
    \lstinputlisting{kode/til-slides/foldbareksempel.sml}
  \end{block}
\end{frame}

\begin{frame}
  \frametitle{Et par eksempler: \texttt{Lazy.t}}

  \begin{block}{}
    \lstinputlisting{kode/til-slides/Lazy.sig}
  \end{block}
\end{frame}


\begin{frame}
  \frametitle{Et par eksempler: \texttt{Lazy.t}}

  \begin{block}{}
    \lstinputlisting{kode/til-slides/Lazy.sml}
  \end{block}
\end{frame}

\begin{frame}[c]
  \frametitle{Det praktiske}
  \begin{itemize}
  \item Oversættere.
  \item Projektfiler.
  \item Biblioteker.
  \end{itemize}
\end{frame}

\begin{frame}
  \frametitle{Det praktiske: Oversættere \quad (1/4)}
  \textbf{Moscow ML} {\scriptsize(\url{http://www.itu.dk/~sestoft/mosml.html})}
  \begin{itemize}
  \item Har I arbejdet med før.
  \item Har en interaktiv fortolker.
  \item Egner sig ikke så godt til programmer som fordeler sig over flere
    filer.
  \item Implementerer ikke hele standardbiblioteket.
  \end{itemize}
\end{frame}

\begin{frame}
  \frametitle{Det praktiske: Oversættere \quad (2/4)}
  \textbf{SML of New Jersey} {\scriptsize(\url{http://www.smlnj.org})}
  \begin{itemize}
  \item Har en interaktiv fortolker.
  \item Er godt dokumenteret.
  \item Understøtter projektfiler (CM).
  \item Har adskillige ekstrabiblioteker.
  \end{itemize}
\end{frame}

\begin{frame}
  \frametitle{Det praktiske: Oversættere \quad (3/4)}
  \textbf{MLKit} {\scriptsize(\url{http://www.it-c.dk/research/mlkit})}
  \begin{itemize}
  \item Har en konfigurerbar spildopsamler.
  \item Understøtter projektfiler (MLB).
  \item Har ingen interaktiv fortolker.
  \end{itemize}
\end{frame}

\begin{frame}
  \frametitle{Det praktiske: Oversættere \quad (4/4)}
  \textbf{MLTon} {\scriptsize(\url{http://mlton.org})}
  \begin{itemize}
  \item Har mange ekstrabiblioteker.
  \item Genererer meget hurtige programmer.
  \item Er godt dokumenteret.
  \item Understøtter projektfiler (MLB).
  \item Oversætter enormt langsomt.
  \item Har ingen interaktiv fortolker.
  \end{itemize}
\end{frame}

\begin{frame}
  \frametitle{Det praktiske: Projektfiler \quad (1/2)}
  \textbf{Compilation Manager}
  {\scriptsize(\url{http://www.smlnj.org/doc/CM})}
  \begin{itemize}
  \item Mange avancerede funktioner.
  \item Godt dokumenteret.
  \item Kun brugt af SML of New Jersey.
  \item Kan være svært at sætte sig ind i.
  \end{itemize}
\end{frame}

\begin{frame}
  \frametitle{Det praktiske: Projektfiler \quad (2/2)}
  \textbf{MLB: ML Basis}
  {\scriptsize(\url{http://mlton.org/MLBasis},
    \url{http://www.it-c.dk/research/mlkit/index.php/ML_Basis_Files})}
  \begin{itemize}
  \item Brugt af flere oversættere (MLTon, MLKit, \ldots).
  \item Ikke så avanceret som CM.
  \item Let at bruge og sætte sig ind i.
  \end{itemize}
  \begin{block}{\texttt{MyFaceTube.mlb}}
    \lstinputlisting[language=c]{kode/persondatabase/MyFaceTube.mlb}
  \end{block}
\end{frame}

\begin{frame}
  \frametitle{Det praktiske: Biblioteker \quad (1/4)}
  \textbf{The Standard ML Basis Library}
  {\scriptsize(\url{http://www.standardml.org/Basis})}
  \begin{block}{}
    SML's standard bibliotek. Lær det godt at kende!
  \end{block}
\end{frame}

\begin{frame}
  \frametitle{Det praktiske: Biblioteker \quad (2/4)}
  \textbf{SMLServer}
  {\scriptsize(\url{http://www.smlserver.org})}
  \begin{block}{}
    Web server plugin til Apache. Skriv websider med SML. Understøtter også ML
    Server Pages.
  \end{block}
\end{frame}

\begin{frame}
  \frametitle{Det praktiske: Biblioteker \quad (3/4)}
  \textbf{MyLib}
  {\scriptsize(\url{http://www.github.com/mortenbp/mylib})}
  \begin{block}{}
    Generisk bibliotek udviklet af Jesper Reenberg og mig selv. Udvider
    standardbiblioteket og implementerer forskellige datastrukturer og andre
    nyttigheder. Er i øjeblikket ikke særligt godt dokumenteret, og ændres
    jævnligt. Brug på eget ansvar!
  \end{block}
\end{frame}

\begin{frame}
  \frametitle{Det praktiske: Biblioteker \quad (4/4)}
  \textbf{MLTon Lib}
  {\scriptsize(\url{http://mlton.org/cgi-bin/viewsvn.cgi/mltonlib})}
  \begin{block}{}
    Udvidelse af standardbiblioteket samt adskillige selvindeholdte
    biblioteker. Svært at gå til og indeholder mange vanvittigt avancerede
    ting. Til gengæld er flere dele dokumenteret i videnskabelige
    artikler. (Kig efter Vesa Karvonen.)
  \end{block}
\end{frame}

\begin{frame}
  \frametitle{Afrunding}

  \begin{itemize}
  \item Minimer nødvendigheden af (implicitte?) konventioner og invarianter
    mellem adskilte dele af systemet.
  \item Find frem til hvilke dele af systemet der har med repræsentation og
    hvilke der har med manipulation at gøre.
  \item Brug projektfiler.
  \item Brug biblioteker (søg på Google --- man bliver overrasket over hvad folk
    har lavet).
  \item Brug en eller anden form for versionskontrol.
  \item Vær ikke bange for at lave ting om (det kommer I alligevel til).
  \end{itemize}
\end{frame}

\begin{frame}
  \frametitle{Afrunding: Litteratur \quad (1/4)}
    Oversættere
    \begin{itemize}
    \item \textbf{MLTon}: (Anbefalet)\\
      {\scriptsize\url{http://mlton.org}}
    \item \textbf{SML of New Jersey}: (Anbefalet)\\
      {\scriptsize\url{http://mlton.org}}
    \item \textbf{Moscow ML}:\\
      {\scriptsize\url{http://www.it-c.dk/research/mlkit}}
    \item \textbf{MLKit}:\\
      {\scriptsize\url{http://www.itu.dk/~sestoft/mosml.html}}
    \end{itemize}
\end{frame}

\begin{frame}
  \frametitle{Afrunding: Litteratur \quad (2/4)}
    Biblioteker
    \begin{itemize}
    \item \textbf{The Standard ML Basis Library}: (Anbefalet)\\
      {\scriptsize\url{http://www.standardml.org/Basis}}
    \item \textbf{SMLServer}:\\
      {\scriptsize\url{http://www.smlserver.org}}
    \item \textbf{MyLib}:\\
      {\scriptsize\url{http://www.github.com/mortenbp/mylib}}
    \item \textbf{MLTon Lib}:\\
      {\scriptsize\url{http://mlton.org/cgi-bin/viewsvn.cgi/mltonlib}}
    \end{itemize}
\end{frame}

\begin{frame}
  \frametitle{Afrunding: Litteratur \quad (3/4)}
    Dokumentation
    \begin{itemize}
    \item \textbf{ML Basis}:\\
      {\scriptsize\url{http://mlton.org/MLBasis},
        \url{http://www.it-c.dk/research/mlkit/index.php/ML_Basis_Files}}
    \item \textbf{Compilation Manager}:\\
      {\scriptsize\url{http://www.smlnj.org/doc/CM}}
    \item \textbf{MLLex}: (Lexer)\\
      {\scriptsize\url{http://mlton.org/pages/Documentation/attachments/mllex.pdf}}
    \item \textbf{MLYacc}: (Parser)\\
      {\scriptsize\url{http://mlton.org/pages/Documentation/attachments/mlyacc.pdf}}
    \end{itemize}
\end{frame}

\begin{frame}
  \frametitle{Afrunding: Litteratur \quad (4/4)}
  Bøger
  \begin{itemize}
  \item \textbf{Purely Functional Data Structures}: (Anbefalet)\\
    \textit{Chris Okasaki}, ISBN 978-0-521-63124-2
  \item \textbf{Introduction to Programming Using SML}:\\
    \textit{Michael R. Hansen} og \textit{Hans Rischel}, ISBN 0-201-39820
  \item \textbf{ML For the Working Programmer}:\\
    \textit{Lawrence C. Paulson}, ISBN 0-521-56543-X
  \item \textbf{The Definition of Standard ML (Revised)}:\\
    \textit{Robin Milner}, \textit{Mads Tofte}, \textit{Robert Harper} og
    \textit{David MacQueen}, ISBN 0-262-63181-4
  \end{itemize}
\end{frame}

\begin{frame}[c]
  \begin{center}
    \Huge{KISS}
  \end{center}
\end{frame}

\end{document}





%%% Local Variables: 
%%% mode: latex
%%% TeX-master: t
%%% End: 
